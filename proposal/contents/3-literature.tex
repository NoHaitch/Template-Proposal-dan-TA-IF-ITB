% ------------------------------------------------------------------- %
%  LITERATURE REVIEW CHAPTER
%  AUTHOR: Raden Francisco Trianto Bratadiningrat (@NoHaitch)
%  DATE: 2025-11-06
% ------------------------------------------------------------------- %

% ----- JUDUL BAB -----
\chapter{Kajian Pustaka}

Bab Kajian Pustaka digunakan untuk mendeskripsikan kajian literatur yang terkait dengan persoalan tugas akhir. Tujuan kajian pustaka adalah:

\begin{enumerate}
    \item Memberikan pemahaman yang cukup kepada pembaca tentang teori atau pekerjaan yang terkait langsung dengan penyelesaian persoalan
    \item Menyampaikan informasi apa saja yang sudah ditulis/dilaporkan oleh pihak lain (peneliti/Tugas Akhir/Tesis) tentang hasil penelitian/pekerjaan mereka yang sama/mirip/erat kaitannya dengan persoalan tugas akhir
    \item Menunjukkan kepada pembaca adanya gap seperti pada rumusan masalah yang memang belum terselesaikan
\end{enumerate}

\section{Contoh Subbab}

Contoh subbab.

\subsection{Contoh Referensi Sitasi}

Referensi diletakan pada format biblitext dalam file `references.bib`

Referensi disitasi menggunakan format APA dengan `cite\{\}`

Publikasi ilmiah sudah berkembang dari abad ke-17, dengan terbitnya \textit{Philosophical Transactions} oleh Royal Society. Sistem publikasi ilmiah terus berevolusi untuk menjawab kebutuhan validasi, standarisasi, dan kolaborasi antar ilmuwan dalam mendokumentasikan serta menyebarluaskan pengetahuan secara formal (\cite{fyfe2022}).

\subsubsection{Contoh Gambar}

Contoh gambar adalah sebagaimana terlihat pada Gambar~\ref{fig:contoh-inline}, 

\begin{center}
    \includegraphics[width=10cm]{images/example.jpg}
    \captionof{figure}{Tahapan konstruksi koleksi retorik kalimat}
    \label{fig:contoh-inline}
\end{center}


\subsubsubsection{Contoh Tabel}

Data dapat dilihat pada Tabel~\ref{tab:contoh-merge} berikut.

\begin{xltabular}{\textwidth}{|c|l|X|}
\caption{Tahapan Proses Publikasi Ilmiah}\label{tab:proses-publikasi} \\

\hline
\rowcolor[HTML]{EFEFEF}
\textbf{No.} & \textbf{Tahapan Proses} & \textbf{Penjelasan} \\
\hline
\endfirsthead
\multicolumn{3}{c}{{\tablename\ \thetable{} Tahapan Proses Publikasi Ilmiah}} \\
\hline
\rowcolor[HTML]{EFEFEF}
\textbf{No.} & \textbf{Tahapan Proses} & \textbf{Penjelasan} \\
\hline
\endhead

\hline
\multicolumn{3}{r}{{Bersambung ke halaman berikutnya}} \\
\endfoot

\hline
\endlastfoot

1 & \textbf{\textit{Manuscript Preparation}} & Penulis menyusun manuskrip sesuai panduan dan standar jurnal tujuan. \\
\hline
2 & \textbf{\textit{Initial Submission}} & Manuskrip dikirimkan oleh penulis ke jurnal untuk dipertimbangkan proses seleksi. \\
\hline
3 & \textbf{\textit{Editorial Screening}} & Editor menilai cakupan, orisinalitas, dan kelayakan dasar manuskrip; manuskrip yang tidak memenuhi kriteria dapat ditolak pada tahap awal (\textit{desk rejection}). \\
\hline
4 & \textbf{\textit{Peer Review Selection}} & Editor memilih dan mengundang penelaah ahli yang relevan. \\
\hline
5 & \textbf{\textit{Peer Review Evaluation}} & Penelaah secara independen menilai kualitas, metodologi, dan kontribusi ilmiah manuskrip. \\
\hline
6 & \textbf{\textit{Reviewer Recommendations}} & Penelaah memberikan rekomendasi penerimaan, revisi, atau penolakan. \\
\hline
7 & \textbf{\textit{Editorial Decision}} & Editor menentukan keputusan akhir berdasarkan hasil penelaahan. \\
\hline
8 & \textbf{\textit{Decision Communication}} & Editor menyampaikan keputusan dan umpan balik kepada penulis. \\
\hline
9 & \textbf{\textit{Author Revisions}} & Penulis melakukan revisi berdasarkan umpan balik yang diberikan. \\
\hline
10 & \textbf{\textit{Resubmission}} & Manuskrip revisi dikirimkan kembali oleh penulis. \\
\hline
11 & \textbf{\textit{Re-review Process}} & Editor dan/atau penelaah menilai ulang revisi manuskrip. \\
\hline
12 & \textbf{\textit{Final Decision}} & Editor membuat keputusan akhir terhadap manuskrip. \\
\hline
13 & \textbf{\textit{Production/Copyediting}} & Tim produksi menyunting bahasa, memformat, dan memeriksa naskah sebelum terbit. \\
\hline
14 & \textbf{\textit{Publication}} & Artikel dipublikasikan secara resmi, daring atau cetak, dengan DOI. \\
\hline
15 & \textbf{\textit{Post-Publication}} & Pembaca dapat memberi umpan balik atau komentar paska-publikasi. \\
\end{xltabular}
