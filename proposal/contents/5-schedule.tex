% ------------------------------------------------------------------- %
%  SCHEDULE CHAPTER
%  AUTHOR: Raden Francisco Trianto Bratadiningrat (@NoHaitch)
%  DATE: 2025-10-28
% ------------------------------------------------------------------- %

% ----- JUDUL BAB -----
\chapter{Rencana Pelaksanaan}

% ----- JADWAL -----
\section{Jadwal}

Jadwal pelaksanaan tugas akhir dapat dilihat pada Tabel~\ref{tab:jadwal-pelaksanaan} berikut.

\begin{table}[!htbp]
    \centering
    \DefaultTableFormatting                
    \caption{Jadwal Pelaksanaan Tugas Akhir}
    \begin{tabularx}{\textwidth}{|c|X|c|}
        \hline
        \rowcolor[HTML]{EFEFEF}
        \textbf{No.} & \textbf{Kegiatan} & \textbf{Waktu Pelaksanaan} \\
        \hline
        \rowcolor[HTML]{FFFFFF}
        1 & Studi Literatur & Januari 2025 \\
        \hline
        \rowcolor[HTML]{F7F7F7}
        2 & Analisis Kebutuhan & Januari 2025 \\
        \hline
        \rowcolor[HTML]{FFFFFF}
        3 & Perancangan Sistem & Januari 2025 \\
        \hline
        \rowcolor[HTML]{F7F7F7}
        4 & Implementasi & Januari 2025 \\
        \hline
        \rowcolor[HTML]{FFFFFF}
        5 & Pengujian & Januari 2025 \\
        \hline
        \rowcolor[HTML]{F7F7F7}
        6 & Dokumentasi & Januari 2025 \\
        \hline
    \end{tabularx}
    \label{tab:jadwal-pelaksanaan}
\end{table}

% ----- RESIKO -----
\section{Risiko}

Terdapat risiko-risiko yang mungkin dihadapi dalam pengerjaan tugas akhir. Risiko-risiko tersebut mencakup risiko teknis, operasional, dan metode yang digunakan. Beberapa risiko tertinggi yang mungkin dihadapi adalah sebagai berikut.

\begin{enumerate}
    \item Risiko teknis: keterlambatan dalam pengembangan perangkat lunak.
    \item Risiko operasional: kurangnya koordinasi antara anggota tim.
    \item Risiko metode: ketidaksesuaian metode yang digunakan dengan kebutuhan proyek.
\end{enumerate}

Rencana mitigasi terhadap risiko-risiko tersebut adalah sebagai berikut.

\begin{enumerate}
    \item Mitigasi risiko teknis: melakukan perencanaan yang matang dan pengujian berkala.
    \item Mitigasi risiko operasional: mengadakan rapat koordinasi secara rutin.
    \item Mitigasi risiko metode: melakukan evaluasi metode sebelum diterapkan.
\end{enumerate}
